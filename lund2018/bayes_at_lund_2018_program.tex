%% LyX 2.2.3 created this file.  For more info, see http://www.lyx.org/.
%% Do not edit unless you really know what you are doing.
\documentclass[english]{article}
\usepackage{mathpazo}
\usepackage[T1]{fontenc}
\usepackage[latin9]{inputenc}
\usepackage[a5paper]{geometry}
\geometry{verbose,tmargin=1.5cm,bmargin=1.5cm,lmargin=1.5cm,rmargin=1.5cm}
\pagestyle{empty}
\usepackage{babel}
\usepackage{graphicx}
\usepackage[unicode=true,
 bookmarks=true,bookmarksnumbered=false,bookmarksopen=false,
 breaklinks=false,pdfborder={0 0 0},pdfborderstyle={},backref=false,colorlinks=false]
 {hyperref}
\hypersetup{pdftitle={Bayes@Lund 2016}}

\makeatletter
%%%%%%%%%%%%%%%%%%%%%%%%%%%%%% Textclass specific LaTeX commands.
\newenvironment{lyxlist}[1]
{\begin{list}{}
{\settowidth{\labelwidth}{#1}
 \setlength{\leftmargin}{\labelwidth}
 \addtolength{\leftmargin}{\labelsep}
 \renewcommand{\makelabel}[1]{##1\hfil}}}
{\end{list}}

\@ifundefined{date}{}{\date{}}
\makeatother

\begin{document}

\title{\includegraphics[width=12cm]{bayes_at_lund_logo_2018}}

\author{\emph{A Mini-conference on Bayesian Methods at Lund University}\\
\emph{12th of April, 2018}\\
\emph{ Lux building, Helgonav�gen 3, Lund University.}\\
\texttt{\emph{\href{https://bayesat.github.io/lund2018/bayes_at_lund_2018.html}{bayesat.github.io/lund2018/bayes\_{}at\_{}lund\_{}2018.html} }}}
\maketitle

\section*{Program}
\begin{lyxlist}{00.00.0000}
\item [{09.10\textendash 10.00}] Welcome and keynote presentation
\item [{~}] $\triangleright$ \emph{Why not to be afraid of priors (too
much)}, Paul-Christian B�rkner, University of M�nster, Department
of Psychology.
\item [{10.00\textendash 10.20}]~
\item [{~}] $\triangleright$ \emph{Bayesian 3D Priors for Brain Imaging},
Per Sid�n, Lindk�ping university, Department of Computer and Information
Science.
\item [{10.20\textendash 10.40}] Coffee break
\item [{10.40\textendash 12.00}] Session: Bayesian Regression Models using
Stan 
\item [{~}] $\triangleright$ \emph{Introducing the brms R package,} Paul-Christian
B�rkner, University of M�nster, Department of Psychology. 
\item [{~}] $\triangleright$ \emph{A hands-on example of Bayesian mixed
models with brms}, Andrey Anikin, Lund University, Cognitive Science. 
\item [{~}] $\triangleright$ \emph{Analyzing an experiment on involuntary
attention using brms}, Antonio Schettino, Ghent University, Department
of Experimental Clinical and Health Psychology. 
\item [{~}] $\triangleright$ \emph{Analyzing orientation behavior in
animals using Stan}, John Kirwan, Lund University, Department of Biology. 
\item [{12.00\textendash 12.45}] Sandwich lunch and mingle in the foyer
\item [{12.45\textendash 13.30}] Keynote presentation
\item [{~}] $\triangleright$ \emph{I know what you ate last summer! \textendash{}
the virtue of Bayesian analysis in food risk assessment}, Jukka Ranta,
Evira Finnish Food Safety Authority, Risk Assessment Unit.
\item [{13.30\textendash 13.50}]~
\item [{~}] $\triangleright$ \emph{Inference in ecology and evolution
beyond generalized linear mixed models}, Reinder Radersma, Lund university,
Department of Biology.
\item [{13.50\textendash 14.00}] Quick break
\item [{14.00\textendash 15.00}] Session: Bayesian hypothesis testing 
\item [{~}] $\triangleright$ \emph{Is there something out there? }, Ullrika
Sahlin, Lund University, Centre for Environmental and Climate Research. 
\item [{~}] $\triangleright$ \emph{Bayes Factors: A \textquoteleft re-volution\textquoteright{}
in psychology}, Geoff Patching, Lund University, Department of Psychology. 
\item [{~}] $\triangleright$ \emph{Sequential Testing with Information
Criteria and Evidence Ratios}, Ladislas Nalborczyk, Univ. Grenoble
Alpes \& Ghent University.
\item [{15.00\textendash 15.30}] Coffee and cake
\item [{15.30\textendash 16.15}] Session: Approximate Bayesian inference 
\item [{~}] $\triangleright$ \emph{What to do when exact Bayes is impossible?
Some tools for approximate Bayesian inference,} Umberto Picchini,
Lund University, Centre for Mathematical Sciences.
\item [{~}] $\triangleright$ \emph{Making the most out of a single datapoint
using Approximate Bayesian inference}, Denis Shepelin, Department
of Computer and Information Science, Link�ping University.
\item [{16.15\textendash 16.30}]~
\item [{~}] $\triangleright$ \emph{How I introduce Bayes to beginners},
Rasmus B��th, King Digital Entertainment, Malm�.
\end{lyxlist}
\newpage{}

\section*{Keynote presentations }

\subsection*{Why not to be afraid of priors (too much)}

\emph{Paul-Christian B�rkner, University of M�nster, Department of
Psychology, }\\
\emph{paul.buerkner@gmail.com }\\
\\
The prior is a key concept in Bayesian statistics that distinguishes
it from most other statistical methods. Historically, it has caused
many resentments against Bayesian statistics in general and remains
a controversial topic to date. In my talk, I want to explain why we
should usually not be too afraid of priors. At the same time, I want
to highlight situations where thinking about priors is mandatory and
incredibly helpful for reliable inference.

\subsection*{I know what you ate last summer! \textendash{} the virtue of Bayesian
analysis in food risk assessment}

\emph{Jukka Ranta, Evira Finnish Food Safety Authority, Risk Assessment
Unit,}\\
\emph{Jukka.Ranta@evira.fi }\\
\\
Evaluation of microbiological and chemical food safety risks typically
involves several linked models. These describe different causal processes
contributing to the consumer risk and can employ several data sets.
When combining evidence, Bayesian modeling is a valuable and flexible
method that can be used to assess, e.g., process control options and
dietary exposure.

\section*{Contributed presentations}

\subsection*{Bayesian 3D Priors for Brain Imaging}

\emph{Per Sid�n, Lindk�ping university, Department of Computer and
Information Science, per.siden@liu.se }\\
\\
This talk discusses the use of Bayesian priors for modeling the spatial
distribution of brain activity from functional magnetic resonance
imaging (fMRI). The data are naturally 4D, consisting of 3D brain
images measured over time, each having hundreds of thousands of data
points. The Bayesian approach is attractive because (1) it gives posterior
probabilities of activation in different brain regions; (2) the amount
of spatial dependence can be inferred from the data and (3) the large
scale computational problem can be efficiently handled using sparse
Gaussian Markov random fields (GMRF) priors. 

\subsection*{Introducing the brms R package}

\emph{Paul-Christian B�rkner, University of M�nster, Department of
Psychology,}\\
\emph{paul.buerkner@gmail.com }\\
\\
This talk with introduce the the brms package which implements Bayesian
multilevel models in R using the probabilistic programming language
Stan. A wide range of distributions and link functions are supported,
allowing users to fit linear, robust linear, binomial, Poisson, survival,
response times, ordinal, quantile, zero-inflated, hurdle, and even
non-linear models all in a multilevel context. In addition, all parameters
of the response distribution can be predicted in order to perform
distributional regression. Prior specifications are flexible and explicitly
encourage users to apply prior distributions that actually reflect
their beliefs. In addition, model fit can easily be assessed and compared
with posterior predictive checks and leave-one-out cross-validation.

\subsection*{A hands-on example of Bayesian mixed models with brms}

\emph{Andrey Anikin, Lund University, Cognitive Science, andrey.anikin@lucs.lu.se
}\\
\\
While the advantages of a Bayesian approach are increasingly acknowledged,
many are discouraged by the steep learning curve and the large amount
of manual coding. In this talk I demonstrate how mixed models, a staple
of modern data analysis, can be easily fit and explored with just
a few lines of code using brms package. Starting with a dataset from
a real experiment, in which listeners judged the authenticity of emotional
vocalizations (a binary outcome), I will go through model specification,
diagnostics, plotting, and reporting the results.

\subsection*{Analyzing an experiment on involuntary attention using brms}

\emph{Antonio Schettino, Ghent University, Department of Experimental
Clinical and Health Psychology, antonio.schettino@ugent.be }\\
\\
It is difficult to suppress the urge to look at your smartphone if
there is a sudden flash on the screen. But would you still look at
the phone if the flash always signals wrong information? We addressed
this question using a visual temporal order judgment task, in which
participants had to judge which of two stimuli appeared first. Some
trials were preceded by a counterproductive exogenous cue, i.e., always
signaling where the second target would appear. Bayesian parameter
estimation and model comparison (using the brms package in R) revealed
that the cue, despite being counterproductive, consistently attracted
attention. 

\subsection*{Analyzing orientation behavior in animals using Stan}

\emph{John Kirwan, Lund University, Department of Biology, john.kirwan@biol.lu.se
}\\
\\
A recurring challenge in behavioral studies of animal senses is the
analysis of angular data, which frequently occurs from tracking orientation
or direction of movement. Analysis and visualization of these data
are fraught with challenges, as they can be easy to misinterpret.
Within the fields of animal sensory ecology and navigation, angular
data has typically been investigated by comparing treatments using
a limited set of significance tests. These approaches are inflexible
and neglect pooling and measures of effect size and uncertainty. To
address this, I have sought to analyse angular behavioral data using
models in the Stan language.

\subsection*{Inference in ecology and evolution beyond generalized linear mixed
models}

\emph{Reinder Radersma, Lund university, Department of Biology, }\\
\emph{reinder.radersma@biol.lu.se}\\
\\
Generalized linear mixed models (GLMMs) are commonly used in many
research fields, including ecology and evolution. GLMMs and the software
for analyzing them offer great flexibility in for instance the number
of variables and error structure, but obviously assume linearity.
The tremendous flexibility of Stan to customize the structure of models
offers a great, but underused potential for investigating more complex
correlational/causality patterns. Here I will showcase 3 studies on
humans, birds and waterfleas, in which I have used Stan to customize
model structures to answer questions in ecology and evolution for
which GLMMs have limited potential.

\subsection*{Is there something out there? }

\emph{Ullrika Sahlin, Lund University, Centre for Environmental and
Climate Research, ullrika.sahlin@gmail.com }\\
\\
I will use this simple question to demonstrate how Bayesian analysis
can be used for risk assessment under sparse information and how it
allows you to incorporate judgement from one or several experts.

\subsection*{Bayes Factors: A \textquoteleft re-volution\textquoteright{} in psychology}

\emph{Geoff Patching, Lund University, Department of Psychology,}\\
\emph{geoffrey.patching@psy.lu.se }\\
\\
In psychology, Bayes Factors (BFs) are being increasingly reported
as a complement to p-values. However, just as the mindless computation
of p-values encourages a simple dichotomization of study results,
the mindless computation of BFs with an overemphasis on hypothesis
testing detracts from more useful approaches to interpreting study
results, such as parameter estimation. BFs may serve to facilitate
the transition from frequentist to Bayesian data analysis, but let
us not repeat mistakes of the past. In this short talk, I shall argue
that Bayesian parameter estimation is preferred with an example from
a recent Master thesis.

\subsection*{Sequential Testing with Information Criteria and Evidence Ratios}

\emph{Ladislas Nalborczyk, Univ. Grenoble Alpes \& Ghent University,}\\
\emph{ladislas.nalborczyk@univ-grenoble-alpes.fr }\\
\\
Sequential testing refers to the process of collecting data until
a predefined level of evidence is reached. Recently, Sch�nbrodt, Wagenmakers,
Zehetleitner, \& Perugini (2017) and Sch�nbrodt \& Wagenmakers (2017)
have introduced the Sequential Bayes Factors (SBF) procedure, in order
to avoid the pitfalls associated with sequential testing in the NHST
framework. The R package ESTER (Nalborczyk, 2017) proposes an alternative
approach that uses evidence ratios based on either Akaike weights
computed from AIC (e.g., Burnham \& Anderson, 2004) or pseudo-BMA
weights computed from the WAIC or the LOO-CV of Bayesian models (Yao,
Simpson, \& Gelman, 2017).

\subsection*{What to do when exact Bayes is impossible? Some tools for approximate
Bayesian inference}

\emph{Umberto Picchini, Lund University, Centre for Mathematical Sciences,}\\
\emph{umberto.picchini@gmail.com }\\
\\
So you have postulated a nice model and now you want to perform Bayesian
inference for the model parameters. Great. The problem is the model
has a non-standard likelihood function.. In other words your model
is sufficiently complex that you can't write the likelihood function
and therefore cannot perform full-fledged Bayesian analysis. This
situation is actually the norm when working with realistic models.
However there are strategies to deal with so-called \textquotedbl{}intractable
likelihoods\textquotedbl{} using \textquotedbl{}likelihood-free inference\textquotedbl{}.
The most important examples of these methods are \textquotedbl{}approximate
Bayesian computation\textquotedbl{} (ABC) and \textquotedbl{}synthetic
likelihoods\textquotedbl{}. I will give an introduction to both methods.

\subsection*{Making the most out of a single datapoint using Approximate Bayesian
inference}

\emph{Denis Shepelin, Department of Computer and Information Science,
Link�ping University, denshe@biosustain.dtu.dk }\\
\\
Bayesian inference is based on evaluation of posterior distribution
via likelihood function. However even for the simplest models sometimes
there is no way for defining likelihood function in analytical way.
In such situations, we still can perform Bayesian reasoning via Approximate
Bayesian Computation techniques. In my talk, I want to provide an
overview on typical ways to perform ABC and corresponding software
packages. To demonstrate flexibility of this method I would like to
present my use case on performing ABC on very complex simulator based
model with 1 data point while still achieving useful results. 

\subsection*{How I introduce Bayes to beginners}

\emph{Rasmus B��th, King Digital Entertainment, rasmus.baath@gmail.com
}\\
\\
Bayesian statistics is a rich a deep topic, but you have to start
somewhere. And there are many places to start, and many ways educators
introduce Bayesian statistics: From probability as personal belief
to just focussing on Markov chain Monte Carlo. In this short talk
I'll give a demonstration of what I think is one good way of introducing
Bayes to beginners.

\newpage{}

\section*{Acknowledgement}

The organizing committee of this year\textquoteright s Bayes@Lund
consisted of Rasmus B��th (King Digital Entertainment, Malm�) and
Ullrika Sahlin (Centre of Environmental and Climate Research, Lund
University). 

We are deeply grateful to the COMPUTE research school for financial
support of this event. We would also like to thank the graduate research
school ClimBEco for helping us bring the invited speakers to Lund
University.\vfill{}
\begin{center}
\includegraphics[width=5cm]{lu_logotype}
\par\end{center}
\end{document}
